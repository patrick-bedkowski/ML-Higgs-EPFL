\section{The Structure of a Paper}
\label{sec:structure-paper}

Scientific papers usually begin with the description of the problem,
justifying why the problem is interesting. Most importantly, it argues
that the problem is still unsolved, or that the current solutions are
unsatisfactory. This leads to the main gist of the paper, which is
``the idea''. The authors then show evidence, using derivations or
experiments, that the idea works. Since science does not occur in a
vacuum, a proper comparison to the current state of the art is often
part of the results. Following these ideas, papers usually have the
following structure:
\begin{description}
\item[Abstract] \ \\
  Short description of the whole paper, to help the
  reader decide whether to read it.
\item[Introduction] \ \\
  Describe your problem and state your
  contributions.
\item[Models and Methods] \ \\
  Describe your idea and how it was implemented to solve
  the problem. Survey the related work, giving credit where credit is
  due.
\item[Results] \ \\
  Show evidence to support your claims made in the
  introduction.
\item[Discussion] \ \\
  Discuss the strengths and weaknesses of your
  approach, based on the results. Point out the implications of your
  novel idea on the application concerned.
\item[Summary] \ \\
  Summarize your contributions in light of the new
  results.
\end{description}